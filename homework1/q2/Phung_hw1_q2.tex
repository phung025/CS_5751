%%%%%%%%%%%%%%%%%%%%%%%%%%%%%%%%%%%%%%%%%
% Structured General Purpose Assignment
% LaTeX Template
%
% This template has been downloaded from:
% http://www.latextemplates.com
%
% Original author:
% Ted Pavlic (http://www.tedpavlic.com)
%
% Note:
% The \lipsum[#] commands throughout this template generate dummy text
% to fill the template out. These commands should all be removed when 
% writing assignment content.
%
%%%%%%%%%%%%%%%%%%%%%%%%%%%%%%%%%%%%%%%%%

%----------------------------------------------------------------------------------------
%	PACKAGES AND OTHER DOCUMENT CONFIGURATIONS
%----------------------------------------------------------------------------------------

\documentclass{article}

\usepackage{fancyhdr} % Required for custom headers
\usepackage{lastpage} % Required to determine the last page for the footer
\usepackage{extramarks} % Required for headers and footers
\usepackage{graphicx} % Required to insert images
\usepackage{lipsum} % Used for inserting dummy 'Lorem ipsum' text into the template
\usepackage{amsmath} % Display math equations
\usepackage{amssymb} % Display math symbol like R, Z

% Margins
\topmargin=-0.45in
\evensidemargin=0in
\oddsidemargin=0in
\textwidth=6.5in
\textheight=9.0in
\headsep=0.25in 

\linespread{1.1} % Line spacing

% Set up the header and footer
\pagestyle{fancy}
\lhead{\hmwkAuthorName} % Top left header
\chead{\hmwkClass\ (\hmwkClassInstructor\ \hmwkClassTime): \hmwkTitle} % Top center header
\rhead{\firstxmark} % Top right header
\lfoot{\lastxmark} % Bottom left footer
\cfoot{} % Bottom center footer
\rfoot{Page\ \thepage\ of\ \pageref{LastPage}} % Bottom right footer
\renewcommand\headrulewidth{0.4pt} % Size of the header rule
\renewcommand\footrulewidth{0.4pt} % Size of the footer rule

\setlength\parindent{0pt} % Removes all indentation from paragraphs

%----------------------------------------------------------------------------------------
%	DOCUMENT STRUCTURE COMMANDS
%	Skip this unless you know what you're doing
%----------------------------------------------------------------------------------------

% Header and footer for when a page split occurs within a problem environment
\newcommand{\enterProblemHeader}[1]{
\nobreak\extramarks{#1}{#1 continued on next page\ldots}\nobreak
\nobreak\extramarks{#1 (continued)}{#1 continued on next page\ldots}\nobreak
}

% Header and footer for when a page split occurs between problem environments
\newcommand{\exitProblemHeader}[1]{
\nobreak\extramarks{#1 (continued)}{#1 continued on next page\ldots}\nobreak
\nobreak\extramarks{#1}{}\nobreak
}

\setcounter{secnumdepth}{0} % Removes default section numbers
\newcounter{homeworkProblemCounter} % Creates a counter to keep track of the number of problems

\newcommand{\homeworkProblemName}{}
\newenvironment{homeworkProblem}[1][Problem \arabic{homeworkProblemCounter}]{ % Makes a new environment called homeworkProblem which takes 1 argument (custom name) but the default is "Problem #"
\stepcounter{homeworkProblemCounter} % Increase counter for number of problems
\renewcommand{\homeworkProblemName}{#1} % Assign \homeworkProblemName the name of the problem
\section{\homeworkProblemName} % Make a section in the document with the custom problem count
\enterProblemHeader{\homeworkProblemName} % Header and footer within the environment
}{
\exitProblemHeader{\homeworkProblemName} % Header and footer after the environment
}

\newcommand{\problemAnswer}[1]{ % Defines the problem answer command with the content as the only argument
\noindent\framebox[\columnwidth][c]{\begin{minipage}{0.98\columnwidth}#1\end{minipage}} % Makes the box around the problem answer and puts the content inside
}

\newcommand{\homeworkSectionName}{}
\newenvironment{homeworkSection}[1]{ % New environment for sections within homework problems, takes 1 argument - the name of the section
\renewcommand{\homeworkSectionName}{#1} % Assign \homeworkSectionName to the name of the section from the environment argument
\subsection{\homeworkSectionName} % Make a subsection with the custom name of the subsection
\enterProblemHeader{\homeworkProblemName\ [\homeworkSectionName]} % Header and footer within the environment
}{
\enterProblemHeader{\homeworkProblemName} % Header and footer after the environment
}
   
%----------------------------------------------------------------------------------------
%	NAME AND CLASS SECTION
%----------------------------------------------------------------------------------------

\newcommand{\hmwkTitle}{Assignment\ \#1} % Assignment title
\newcommand{\hmwkDueDate}{January\ 23,\ 2018} % Due date
\newcommand{\hmwkClass}{CS\ 5751} % Course/class
\newcommand{\hmwkClassTime}{11:00am} % Class/lecture time
\newcommand{\hmwkClassInstructor}{Eleazar} % Teacher/lecturer last name
\newcommand{\hmwkAuthorName}{Nam Phung} % Your name

%----------------------------------------------------------------------------------------
%	TITLE PAGE
%----------------------------------------------------------------------------------------

\title{
\vspace{2in}
\textmd{\textbf{\hmwkClass:\ \hmwkTitle}}\\
\normalsize\vspace{0.1in}\small{Due\ on\ \hmwkDueDate}\\
\vspace{0.1in}\large{\textit{\hmwkClassInstructor\ \hmwkClassTime}}
\vspace{3in}
}

\author{\textbf{\hmwkAuthorName}}
\date{} % Insert date here if you want it to appear below your name

%----------------------------------------------------------------------------------------

\begin{document}

\maketitle

%----------------------------------------------------------------------------------------
%	TABLE OF CONTENTS
%----------------------------------------------------------------------------------------

%\setcounter{tocdepth}{1} % Uncomment this line if you don't want subsections listed in the ToC

\newpage
%\tableofcontents
%\newpage

%----------------------------------------------------------------------------------------
%	PROBLEM 1
%----------------------------------------------------------------------------------------

\begin{homeworkProblem}[Exercise \#\arabic{homeworkProblemCounter}] % Custom section title
% Question
Consider the function $z=f(x,y)=x^2+y^2$. Now, write a pdf file named yourlastname\_hw1\_q2.pdf containing the answers to the following tasks:

%--------------------------------------------

\begin{homeworkSection}{(1)} % Section within problem
	% Question
	Compute the gradient $\nabla f(x,y)$. For this task you need to explain every step of your computation of the gradient. You cannot simply write the vector.\vspace{10pt} 
	
	\problemAnswer{ % Answer
		We have $f(x,y)=x^2+y^2$. Since $\frac{\partial f}{\partial x}=2x$ and $\frac{\partial f}{\partial y}=2y$, we will have the following:
		\begin{align*}
		\nabla f(x,y)
		= \Big(\frac{\partial f}{\partial x}, \frac{\partial f}{\partial y}\Big)
		= (2x,2y)
		\end{align*}
	
	Therefore, the gradient of $f(x,y)=(2x,2y)$.
}
\end{homeworkSection}

%--------------------------------------------

\begin{homeworkSection}{(2)} % Section within problem
	% Question
	Compute the Hessian of $\nabla f(x,y)$. For this task you need to explain every step of your computation of the Hessian. You cannot simply write it the matrix.\vspace{10pt}
	
	\problemAnswer{ % Answer
		We have the function $f(x,y)=x^2+y^2$. In part 1, we know that $\frac{\partial f}{\partial x}=2x$ and $\frac{\partial f}{\partial y}=2y$.  Since $f:\mathbb{R}^2\rightarrow \mathbb{R}$, the Hessian matrix of $f$ is a square matrix with a dimension $2\times 2$, that is:
		
		\begin{align*}
		\nabla ^2 f &=
		\begin{bmatrix}
			\frac{\partial ^2 f}{\partial x^2} & \frac{\partial ^2 f}{\partial xy} \\
			\frac{\partial ^2 f}{\partial yx} & \frac{\partial ^2 f}{\partial y^2} \\ 
		\end{bmatrix} \\
		&= 
		\begin{bmatrix}
		\frac{\partial}{\partial x}2x & \frac{\partial}{\partial y}2x \\
		\frac{\partial}{\partial x}2y & \frac{\partial}{\partial y}2y \\ 
		\end{bmatrix} \\
		&= 
		\begin{bmatrix}
		2 & 0 \\
		0 & 2 \\
		\end{bmatrix}
		\end{align*}
	}
\end{homeworkSection}

%--------------------------------------------

\begin{homeworkSection}{(3)} % Section within problem
	% Question
	Write the second order Taylor series expansion for $f(x,y)$ around $x_0=0$. Explain your answer.\vspace{10pt} 
	
	\problemAnswer{ % Answer
		We have the Taylor series expansion around $x_0=0$ for $f:\mathbb{R}^2\rightarrow \mathbb{R}$ as follows:
		\begin{align*}
			f(x_0+h)=f(x_0)+\Big(\nabla f(x_0)\Big)h + \frac{1}{2} \Big(h^T\Big) \Big[\Big(\nabla^2f(x_0)\Big) h\Big]
		\end{align*}
		We know that $x_0$ and $h$ are vectors with the following values: $x_0=\begin{bmatrix}0 \\ 0\end{bmatrix}$	and $h=\begin{bmatrix}h_0 \\ h_1\end{bmatrix}$. From section 1 and 2 above, we also know that the gradient $\nabla f=(2x,2y)$ and the Hessian of $f$ is $\nabla^2 f=\begin{bmatrix}
		2 & 0 \\
		0 & 2 \\
		\end{bmatrix}$. Therefore, the Taylor series expansion can be written as follows:
		\begin{align*}
			f(x_0+h) &= f(x_0)+\Big(\nabla f(x_0)\Big)h + \Big(\frac{1}{2}\Big) \big(h^T\big) \Bigg[\Big(\nabla^2f(x_0)\Big) \big(h\big)\Bigg] \\
			&= f(0,0)+\Big(\nabla f(0,0)\Big)h + \Big(\frac{1}{2}\Big) \big(h^T\big) \Bigg[\Big(\nabla^2f(0,0)\Big) \big(h\big)\Bigg] \\
			&= 0 + \begin{bmatrix}0 & 0\end{bmatrix} \begin{bmatrix}h_0 \\ h_1\end{bmatrix} + \frac{1}{2} \begin{bmatrix}h_0 & h_1\end{bmatrix} \Bigg(\begin{bmatrix}
			2 & 0 \\
			0 & 2 \\
			\end{bmatrix} \begin{bmatrix}h_0 \\ h_1\end{bmatrix}\Bigg) \\
			&= \begin{bmatrix}
			\frac{h_0}{2} & \frac{h_1}{2} 
			\end{bmatrix}
			\begin{bmatrix}
			2h_0 \\ 2h_1
			\end{bmatrix} \\
			&= h^2_0 + h^2_1
		\end{align*}
	}
\end{homeworkSection}

%--------------------------------------------

\begin{homeworkSection}{(4)} % Section within problem
	% Question
	Consider Figure 2, which is a plot of $f(x,y)$. What do you observe in this plot? In which direction does the gradient point? Why does this happen?\vspace{10pt} 
	
	\problemAnswer{ % Answer
		The plot display a surface of a convex function $f(x,y)=x^2+y^2$. The color of the surface indicates the value $z$ ranging from blue (small value) to red (big value). The arrows on the z-surface is vector field indicating the gradient of $f(x,y)$ at different coordinates $(x,y)$. At the origin $(x,y)=(0,0)$ is the global minimum of this convex function. The gradient of this function is diverging from the origin $(0,0)$. This is because this surface has only 1 minimum at $(x,y)=(0,0)$, as we move further away from this point, the slope of the surface becomes higher, which means the gradient will increase.
		}
\end{homeworkSection}

%--------------------------------------------

\end{homeworkProblem}


%----------------------------------------------------------------------------------------

\end{document}

