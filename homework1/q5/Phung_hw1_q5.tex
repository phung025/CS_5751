%%%%%%%%%%%%%%%%%%%%%%%%%%%%%%%%%%%%%%%%%
% Structured General Purpose Assignment
% LaTeX Template
%
% This template has been downloaded from:
% http://www.latextemplates.com
%
% Original author:
% Ted Pavlic (http://www.tedpavlic.com)
%
% Note:
% The \lipsum[#] commands throughout this template generate dummy text
% to fill the template out. These commands should all be removed when 
% writing assignment content.
%
%%%%%%%%%%%%%%%%%%%%%%%%%%%%%%%%%%%%%%%%%

%----------------------------------------------------------------------------------------
%	PACKAGES AND OTHER DOCUMENT CONFIGURATIONS
%----------------------------------------------------------------------------------------

\documentclass{article}

\usepackage{fancyhdr} % Required for custom headers
\usepackage{lastpage} % Required to determine the last page for the footer
\usepackage{extramarks} % Required for headers and footers
\usepackage{graphicx} % Required to insert images
\usepackage{lipsum} % Used for inserting dummy 'Lorem ipsum' text into the template
\usepackage{amsmath} % Display math equations
\usepackage{amssymb} % Display math symbol like R, Z

% Margins
\topmargin=-0.45in
\evensidemargin=0in
\oddsidemargin=0in
\textwidth=6.5in
\textheight=9.0in
\headsep=0.25in 

\linespread{1.1} % Line spacing

% Set up the header and footer
\pagestyle{fancy}
\lhead{\hmwkAuthorName} % Top left header
\chead{\hmwkClass\ (\hmwkClassInstructor\ \hmwkClassTime): \hmwkTitle} % Top center header
\rhead{\firstxmark} % Top right header
\lfoot{\lastxmark} % Bottom left footer
\cfoot{} % Bottom center footer
\rfoot{Page\ \thepage\ of\ \pageref{LastPage}} % Bottom right footer
\renewcommand\headrulewidth{0.4pt} % Size of the header rule
\renewcommand\footrulewidth{0.4pt} % Size of the footer rule

\setlength\parindent{0pt} % Removes all indentation from paragraphs

%----------------------------------------------------------------------------------------
%	DOCUMENT STRUCTURE COMMANDS
%	Skip this unless you know what you're doing
%----------------------------------------------------------------------------------------

% Header and footer for when a page split occurs within a problem environment
\newcommand{\enterProblemHeader}[1]{
\nobreak\extramarks{#1}{#1 continued on next page\ldots}\nobreak
\nobreak\extramarks{#1 (continued)}{#1 continued on next page\ldots}\nobreak
}

% Header and footer for when a page split occurs between problem environments
\newcommand{\exitProblemHeader}[1]{
\nobreak\extramarks{#1 (continued)}{#1 continued on next page\ldots}\nobreak
\nobreak\extramarks{#1}{}\nobreak
}

\setcounter{secnumdepth}{0} % Removes default section numbers
\newcounter{homeworkProblemCounter} % Creates a counter to keep track of the number of problems

\newcommand{\homeworkProblemName}{}
\newenvironment{homeworkProblem}[1][Problem \arabic{homeworkProblemCounter}]{ % Makes a new environment called homeworkProblem which takes 1 argument (custom name) but the default is "Problem #"
\stepcounter{homeworkProblemCounter} % Increase counter for number of problems
\renewcommand{\homeworkProblemName}{#1} % Assign \homeworkProblemName the name of the problem
\section{\homeworkProblemName} % Make a section in the document with the custom problem count
\enterProblemHeader{\homeworkProblemName} % Header and footer within the environment
}{
\exitProblemHeader{\homeworkProblemName} % Header and footer after the environment
}

\newcommand{\problemAnswer}[1]{ % Defines the problem answer command with the content as the only argument
\noindent\framebox[\columnwidth][c]{\begin{minipage}{0.98\columnwidth}#1\end{minipage}} % Makes the box around the problem answer and puts the content inside
}

\newcommand{\homeworkSectionName}{}
\newenvironment{homeworkSection}[1]{ % New environment for sections within homework problems, takes 1 argument - the name of the section
\renewcommand{\homeworkSectionName}{#1} % Assign \homeworkSectionName to the name of the section from the environment argument
\subsection{\homeworkSectionName} % Make a subsection with the custom name of the subsection
\enterProblemHeader{\homeworkProblemName\ [\homeworkSectionName]} % Header and footer within the environment
}{
\enterProblemHeader{\homeworkProblemName} % Header and footer after the environment
}
   
%----------------------------------------------------------------------------------------
%	NAME AND CLASS SECTION
%----------------------------------------------------------------------------------------

\newcommand{\hmwkTitle}{Assignment\ \#1} % Assignment title
\newcommand{\hmwkDueDate}{January\ 23,\ 2018} % Due date
\newcommand{\hmwkClass}{CS\ 5751} % Course/class
\newcommand{\hmwkClassTime}{11:00am} % Class/lecture time
\newcommand{\hmwkClassInstructor}{Eleazar} % Teacher/lecturer last name
\newcommand{\hmwkAuthorName}{Nam Phung} % Your name

%----------------------------------------------------------------------------------------
%	TITLE PAGE
%----------------------------------------------------------------------------------------

\title{
\vspace{2in}
\textmd{\textbf{\hmwkClass:\ \hmwkTitle}}\\
\normalsize\vspace{0.1in}\small{Due\ on\ \hmwkDueDate}\\
\vspace{0.1in}\large{\textit{\hmwkClassInstructor\ \hmwkClassTime}}
\vspace{3in}
}

\author{\textbf{\hmwkAuthorName}}
\date{} % Insert date here if you want it to appear below your name

%----------------------------------------------------------------------------------------

\begin{document}

\maketitle

%----------------------------------------------------------------------------------------
%	TABLE OF CONTENTS
%----------------------------------------------------------------------------------------

%\setcounter{tocdepth}{1} % Uncomment this line if you don't want subsections listed in the ToC

\newpage
%\tableofcontents
%\newpage

%----------------------------------------------------------------------------------------
%	PROBLEM 1
%----------------------------------------------------------------------------------------

\begin{homeworkProblem}[Exercise \#5] % Custom section title
% Question
Suppose we have an unfair die such that Pr(1) = 0.2, Pr(2) = 0.3, Pr(3) = 0.1, Pr(4) = 0.1, Pr(5) = 0.1, Pr(6) = 0.2. Now do the following tasks. Note that you cannot simply write the value; you need to write the complete formula and then solve it step by step:

%--------------------------------------------

\begin{homeworkSection}{(1)} % Section within problem
	% Question
	What is the expected value of the face on which the die will land?\vspace{10pt} 
	
	\problemAnswer{ % Answer
		We have the expected value of a discrete random variables defined as follows:
		\begin{align*}
		E(\mathbf{X}) &= \sum_{i=1}^{n} x_i Pr(x_i)
		\end{align*}
		Given an unfair die with probabilities defined above, we can compute the expected value of the random variable as follows:
		\begin{align*}
		E(\mathbf{X}) &= \sum_{i=1}^{n} x_i Pr(x_i) \\
		&= \sum_{i=1}^{6} x_i Pr(x_i) \\
		&= (1)Pr(1) + (2)Pr(2) + (3)Pr(3) + (4)Pr(4) + (5)Pr(5) + (6)Pr(6) \\
		&= 0.2 + (2)(0.3) + (3)(0.1) + (4)(0.1) + (5)(0.1) + (6)(0.2) \\
		&= 0.2 + 0.6 + 0.3 + 0.4 + 0.5 + 1.2 \\ 
		&= 3.2 \\
		\end{align*}
}
\end{homeworkSection}

%--------------------------------------------

\begin{homeworkSection}{(2)} % Section within problem
	% Question
	What is the variance of the value of the face on which the die will land?\vspace{10pt}
	
	\problemAnswer{ % Answer
		We have the variance of a discrete random variables defined as follows:
		\begin{align*}
		Var(\mathbf{X}) &= \sum_{i=1}^{k} [x_i-E(\mathbf{X})]^2Pr(x_i) \\
		\end{align*}
		Given an unfair die with probabilities defined above, we can compute the variance of the random variables as follows:
		\begin{align*}
		Var(\mathbf{X}) &= \sum_{i=1}^{k} [x_i-E(\mathbf{X})]^2Pr(x_i) \\
		&= \sum_{i=1}^{6} [x_i-E(\mathbf{X})]^2Pr(x_i) \\
		&= (1-3.2)^2(0.2) + (2-3.2)^2(0.3) + (3-3.2)^2(0.1) + (4-3.2)^2(0.1) + \\ &\quad\,\,(5-3.2)^2(0.1) + (6-3.2)^2(0.2) \\ 
		&= 0.968 + 0.432 + 0.004 + 0.064 + 0.324 + 1.568 \\
		&= 3.36 \\ 
		\end{align*}
	}
\end{homeworkSection}

%--------------------------------------------

\begin{homeworkSection}{(3)} % Section within problem
	% Question
	Suppose that you throw that same die, but you can’t see where it landed.
	Someone else tells you that the top face of the die is even number. What is the probability that the top face is “2”? And the probability that it’s “4”? And the probability that it’s “2” or “4”?\vspace{10pt} 
	
	\problemAnswer{ % Answer
		The probability that the top face is 2 given that the top face of the die is even number can be computed as follows:
		\begin{align*}
		Pr(\mathbf{X}=2 | \mathbf{X}\,\,is\,\,even) &= \frac{Pr( \mathbf{X}\,\,is\,\,even | \mathbf{X}=2) Pr(\mathbf{X}=2)}{Pr(\mathbf{X}\,\,is\,\,even)} \\
		&= \frac{Pr( \mathbf{X}\,\,is\,\,even | \mathbf{X}=2) Pr(\mathbf{X}=2)}{Pr(\mathbf{X}=2)+Pr(\mathbf{X}=4)+Pr(\mathbf{X}=6)} \\
		&= \frac{(1)(0.3)}{0.3+0.1+0.2} \\
		&= 0.5 \\
		\end{align*}
		The probability that the top face is 4 given that the top face of the die is even number can be computed as follows:
		\begin{align*}
		Pr(\mathbf{X}=4| \mathbf{X}\,\,is\,\,even) &= \frac{Pr( \mathbf{X}\,\,is\,\,even | \mathbf{X}=4) Pr(\mathbf{X}=4)}{Pr(\mathbf{X}\,\,is\,\,even)} \\
		&= \frac{(1)(0.1)}{0.6} \\
		&= 0.167 \\
		\end{align*}
		The probability that the top face is 4 or 2 given that it's even number is:
		\begin{align*}
		Pr(\mathbf{X}=4\,\,or\,\,2| \mathbf{X}\,\,is\,\,even) &= \frac{Pr( \mathbf{X}\,\,is\,\,even | \mathbf{X}=4\,\,or\,\,2) Pr(\mathbf{X}=4\,\,or\,\,2)}{Pr(\mathbf{X}\,\,is\,\,even)} \\
		&= \frac{(1)(0.1+0.3)}{0.6} \\
		&= 0.667 \\
		\end{align*}
	}
\end{homeworkSection}

%--------------------------------------------

\end{homeworkProblem}


%----------------------------------------------------------------------------------------

\end{document}

